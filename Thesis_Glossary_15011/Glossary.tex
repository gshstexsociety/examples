% !TeX spellcheck = en_US
\documentclass[12pt]{article}
\usepackage[left=30mm,right=30mm,top=35mm,bottom=30mm]{geometry}
\usepackage{amsmath} % math
\usepackage{amssymb} % math
\usepackage{graphicx} % to use \includegraphics{}
\usepackage{diagbox} % to make tables
\usepackage{multirow}
\usepackage{caption}
\usepackage{subcaption}
\usepackage{bm, array, tabularx, pifont}
\usepackage{apacite}
\usepackage{longtable}
\title{Thesis Glossary}
\author{Steven}
\begin{document}
	\maketitle
\section{Introduction}
	\begin{longtable}{|p{0.5\textwidth}|p{0.5\textwidth}|}
	\hline
	\multicolumn{1}{|c|}{from the textbook}&\multicolumn{1}{c|}{from the articles}\\
	\hline
	\ding{172} p.34 Explaining significance\newline \textbf{fundamental issue}\newline Explaining the mechanisms of high-temperature superconductors has been a \emph{fundamental issue} after BCS theory proposed in 20th century.& \ding{174} Problem \newline \textbf{ambiguous} \newline For a wave packet with a spread in wavenumber $k$, some \emph{ambiguity} arises in the values of the phase and group velocities because of the spread in $k$, but, for narrow packets in $k$ space, the uncertainties in these values are small \cite{intro1}. \\
	\hline
	\ding{173} p.35 Verbs to present current research\newline \textbf{prove}\newline This paper \emph{proves} the Riemann-Zeta hypothesis. & \ding{173} Verbs used to present previous research\newline \textbf{indicate} \newline With the increased current, investigations \emph{indicated} that wear is associated with the intensification of the abrasive properties of the metal counterbody surface \cite{intro2}.\\
	\hline
	\ding{174} p.37 Problem\newline \textbf{computationally demanding, an alternative approach}\newline Since the simulation methods in the previous researches are \emph{computationally demanding}, \emph{an alternative approach} was needed for this paper.& \ding{175} The present work \newline \textbf{propose, discuss, this paper} \newline \emph{This paper proposes} and \emph{discusses} a definition of internal energy \cite{intro3}.\\
	\hline
	\end{longtable}

\section{Methodology}

	\begin{longtable}{|p{0.5\textwidth}|p{0.5\textwidth}|}
		\hline
		\multicolumn{1}{|c|}{from the textbook}&\multicolumn{1}{c|}{from the articles}\\
		\hline
		\ding{172} p.77 Provide a general overview of the methods\newline \textbf{all of, experiments, be carried out}\newline \emph{All of} the \emph{experiments were carried out} at room temperature.& \ding{172} Give the source of materials used \newline \textbf{be provided by} \newline The multiwalled carbon nanotubes used in this work \emph{were provided by} Shenzhen Nanotech Port Co. Ltd \cite{intro2}.\\
		\hline
		\ding{174} p.80 Provide specific details about methods\newline \textbf{be measured}\newline The maximum and the minimum length of the brush \emph{were measured}. & \ding{174} Provide specific details about method \ding{175} Justify choices made\newline \textbf{in order to, improve, treat} \newline \emph{In order to} increase the surface roughness to \emph{improve} the interfacial strength and the dispersion, carbon nanotubes were first subjected to an oxidation \emph{treatment} in the mixture of nitric acid and vitriolic \cite{intro2}.\\
		\hline
		\ding{175} p.82 Justify choices made\newline \textbf{in an attempt to}\newline This experimental conditions were chosen \emph{in an attempt to} obtain the friction coefficient value as close as possible to the actual condition.& \ding{173} Supply essential background information \newline \textbf{be embedded} \newline A flexible pure copper wire of 0.5 mm$^2$ cross-sectional area \emph{was embedded} in each brush at 5 mm from the brush's sliding surface to give the average contact voltage drop of brush. \cite{intro2}.\\
		\hline
	\end{longtable}


\section{Results}
	\begin{longtable}{|p{0.5\textwidth}|p{0.5\textwidth}|}
		\hline
		\multicolumn{1}{|c|}{from the textbook}&\multicolumn{1}{c|}{from the articles}\\
		\hline
		\ding{174} p.139 Invitation to view results\newline \textbf{as illustrated by Fig. 1}\newline \emph{As illustrated by Fig. 1}, black stripes were observed on the brush along the arc of the commutator.& \ding{174} Invitations to view results \newline \textbf{from Fig. 1 it can be seen that} \newline \emph{From Fig.4 we can see that} the friction coefficients decreased from initial values of 0.48 to about 0.25-0.28 (without current) or 0.34-0.37 (with current) \cite{intro2}.\\
		\hline
		\ding{175} p.140 Specific results in detail\newline \textbf{decrease, noticeably}\newline The wear rate \emph{decreased noticeably} as wear progresses and the value of $\alpha$ increases. & \ding{175} Specific results in detail\newline \textbf{important} \newline The ``thermal shock'' arising as a result of Joule heat release on the contact spot was another \emph{important} factor leading to intensification of the wear of the brush under the action of an electric current. \cite{intro2}.\\
		\hline
		\ding{177} p.144 Problems with results\newline \textbf{not always accurate, hard to control}\newline The measurements of the length were \emph{not always accurate}, since the bottom face of the brush was not always horizontal and this effect was \emph{hard to control}.& \ding{176} Comparisons with other results \newline \textbf{confirm} \newline As already found by Prasad et al. and \emph{confirmed}
		here, both solutions are characterized by high contact pressure
		at the leading edge dropping to zero at the trailing edge, suggesting a lift (separation) at that corner \cite{res3}.\\
		\hline
	\end{longtable}


\section{Discussion}

	\begin{longtable}{|p{0.5\textwidth}|p{0.5\textwidth}|}
		\hline
		\multicolumn{1}{|c|}{from the textbook}&\multicolumn{1}{c|}{from the articles}\\
		\hline
		\ding{175} p.188 Mapping\newline \textbf{consistent with}\newline The nearly linear wear behavior of the brush was \emph{consistent with} the previous researches.& \ding{174} Refining the implications \newline \textbf{indicate} \newline The test results \emph{indicate} an initial high wear-rate, which gradually reduces \cite{dis1}.\\
		\hline
		\ding{176} p.190 Contribution\newline \textbf{improve}\newline This paper \emph{improves} the way to predict wear behavior of the brush by considering the structure of the motor.& \ding{177} Current and future research\newline \textbf{future works may} \newline It is suggested that \emph{future works may} explore the idea of using
		the contact modulus, a composite value obtained by combining
		elastic moduli and poisson’s ratios of both material in contact,
		rather then the approach of this paper \cite{res3}.\\
		\hline
		\ding{178} p.193 Applications\newline \textbf{utilize}\newline The result of the paper can be \emph{utilized} for predicting wear life of the brush of the DC motor.& \ding{175} Mapping \newline \textbf{in agreement with} \newline The results for the specific problems considered here are \emph{in full agreement with} the general predictions \cite{dis3}.\\
		\hline
	\end{longtable}


\bibliography{ref}
\bibliographystyle{apacite}
\end{document}